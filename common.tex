%%% common.tex
%
% This file  the example usage, writing suggestions, best practices, etc.
% it is included inside paper-acl.tex, paper-jmlr.tex, etc.
% Like stroop.sty, this file is based on
%%% NERT lab (http://nert.georgetown.edu/) stylesheet
%%% Created by Nathan Schneider
%%% Feel free to use, modify, and share.
%%% https://www.overleaf.com/project/5db4fd68be57c00001336e73

\section{Introduction}
This is an example document produced with a stylesheet from
ACL-organized conferences (ACL, NAACL, EACL, AACL, EMNLP, and
colocated conferences/workshops).
The customizations in stroop.sty (activated by \verb|\usepackage{stroop}| above)
will make your life easier and your paper more attractive.

\section{Getting started}

\begin{enumerate}
    \item Find the submission instructions and stylesheet for the venue you are submitting to. It should be linked from the call for papers or available as an Overleaf template.
    \item This template uses acl2020.sty. Substitute the venue-appropriate file, and update the line \verb|\usepackage[hyperref]{acl2020}| above to match.
    \item Double-check the paper size: some venues require Letter, others A4.
    Specify above in the \verb|\documentclass| command.
    \item Rename mypaper.tex and mypaper.bib to use a short, memorable identifier based on the topic of the paper so it is easy to search for the file on your computer. E.g., for your paper ``COOKIE: Corpus Of Ontological Knowledge for Information Extraction'', you might use cookiecorpus.tex and \mbox{cookiecorpus.bib}. Update \verb|\bibliography{mypaper}| to point to the .bib file.
    \item Author names under the title will be hidden until \verb|\aclfinalcopy| is uncommented.
    \item If using Overleaf, consider activating the built-in Git repository for your paper. Clone the repository to your machine so you can edit offline. Every time you \verb|git pull| it will generate and download a commit for all new changes made in the web interface. Even if you are not editing locally, it is good to make frequent backups in case something goes awry.
\end{enumerate}

\section{Working in drafts}

\paragraph{Notes.}
In stroop.sty there are commands for you to leave notes with your initials:
\vn{Vlad},
\gc{Gon\c{c}alo}
At submission time, you can hide them by uncommenting
\verb|\renewcommand{\nertcomment}[4]{\unskip}| before the start of the document.

You can also use Overleaf's commenting feature, but these comments will not be printed in the draft.

\paragraph{Document versions.}
There are macros you can use to turn certain content on or off
depending on a particular stage in the lifecycle of the paper:
\begin{itemize}
    \item \verb|\draftversion{...}| for content that goes in internal drafts, \verb|\subversion{...}| for the official submission, \verb|\finalversion{...}| for the final version
    \item \verb|\anonversion{...}| for when the paper is anonymized,
    \verb|\nonanonversion{...}| for when it isn't
    \item Length alternatives for working within a page limit:\\ \verb|\shortversion{...}|, \verb|\longversion{...}|, \verb|\shortlong{short...}{long...}|, \verb|\considercutting{...}|
\end{itemize}
Edit the macro definitions in \verb|stroop.sty| to control which will display
their contents and which will be hidden
(they are independent of \verb|\aclfinalcopy|).

\section{Bibliography and citations}\label{sec:bib}

\paragraph{Bibliography.}
I recommend creating a fresh .bib file specifically for each paper you write.
If you curate citation info in a reference manager, you can export the relevant
citations as Bibtex
(Zotero users, see \url{https://gist.github.com/nschneid/2257875} for customizing
how Bibtex entries are created).
The \href{https://www.aclweb.org/anthology/}{ACL Anthology} generally has high-quality .bib entries.
Entries should include URLs where possible.

By default, only papers cited at least once will appear under \textbf{References}.\footnote{\label{fn:emptybib}If your paper has an empty .bib file or no citations, compiling will trigger the error \texttt{Something's wrong-{}-perhaps a missing \textbackslash item}.}
If you want to circumvent this and force rendering of all entries in the
included bibfile, use \verb|\nocite{*}|, and to include only specific bib keys
use \verb|\nocite{key1,key2}|. If you think you need to do this, think twice --
this should only be used, \eg, if you want to generate a list of all your
papers.

Pay special attention to \textbf{capitalization in titles}:
by default, the ACL stylesheet will lowercase all characters except the first in the title.

\paragraph{Citations.}
For ACL, ICML, JMLR, and many others,
citation commands are based on the \href{http://tug.ctan.org/macros/latex/contrib/natbib/natbib.pdf}{\texttt{natbib}} package, including
\verb|\citep| for parenthesized citations (authors in parentheses),
\verb|\citet| for textual citations (authors before parentheses),
\verb|\citeposs| for textual citations with a possessive author, and
\verb|\citealp| for citations without any parentheses.
Use capitalized equivalents (e.g., \verb|\Citet|) at the beginning
of a sentence. Optional square brackets specify extra material to put at the beginning or end within the parentheses.
For example:

\begin{quote}
\Citet{blodgett-18} provided a technique that has been adapted in subsequent work on English \citep{schneider-18,prange-19} and extended to other languages \citep[e.g.,][]{zhu-19}.
Current practice is detailed in the latest guidelines \citep[pp.~77--78]{snacs}.
\Citeposs{manning-19} system adopts a different approach.
\end{quote}

\section{Cross-references}\label{sec:xref}

Units of your document---including sections, numbered linguistic examples,
tables, and figures---can be labeled for cross-referencing.
This is advisable in case the structure of your document changes.
It is recommended to use consistent prefixes in your labels:
\verb|\label{fig:myfigure}|, \verb|\label{tab:mytable}|,
\verb|\label{sec:mysection}|, \verb|\label{ex:myexample}|,
\verb|\label{fn:myfootnote}|, etc.\footnote{\label{fn:captions}Note that
table and figure labels must go inside or after the \texttt{\textbackslash caption} command.}

{\bfseries Please use the \verb|\cref| command (based on the \href{http://tug.ctan.org/macros/latex/contrib/cleveref/cleveref.pdf}{\texttt{cleveref}} package) for cross-references.}
It will automatically display the number with the word ``figure'' or ``table'',
the section symbol (\S), or other formatting as appropriate.
It supports a comma-separated list of multiple cross-references of the same type.
Capitalize as \verb|\Cref| at the beginning of a sentence.
For example:

\begin{quote}
\Cref{fn:captions} provides an important addendum. Cross-references and linguistic examples are discussed herein (\cref{sec:xref,sec:ling}).
\end{quote}

\section{Linguistic examples}\label{sec:ling}

For most purposes, the \href{http://texdoc.net/texmf-dist/doc/latex/linguex/linguex-doc.pdf}{\texttt{linguex}} package provides the simplest commands
for numbered linguistic examples. They look like this:

\ex.\label{ex:first} Here is an example.

\ex.\label{ex:second} Here is another one.
    \a. It has a subpart.
    \b.\label{ex:subpart2} And a subsequent subpart.
        \a.\label{ex:subsubpart} with its own subpart
        \z.
    \b.\label{ex:subpart3} And so on.
    \b.\label{ex:subpart4} And so forth.

The cross-referencing commands (\cref{sec:xref}) will work: \cref{ex:second}, \cref{ex:subsubpart}, \cref{ex:subpart2,ex:subpart3,ex:subpart4}.
Glosses are also supported; see the \href{http://mirrors.ctan.org/macros/latex/contrib/linguex/doc/linguex-doc.pdf}{package documentation} for details.

stroop.sty defines a few macros for linguistic notation:
\sst{Supersense}, \psst{SNACSSupersense}, \rf{Role}{Function};
boxed numbers \svar{1}, \SVAR{2}; etc.
It also provides dingbats (\chk~\xxx) and special character aliases (\backtick~\tat).

\section{Style}

There are many places you can go for \LaTeX{} typesetting tips.
A few guidelines of particular note are listed below.

\subsection{Text}

\begin{itemize}
    \item Look carefully at ``quotes'' to ensure they're curly and going in the right direction. ''BAD"
    \item With \texttt{fnpct}, all footnotes appear \emph{after} post-word punctuation like periods and commas.
    It feels illogical,\footnote{not logical}
    but the other way looks ugly.
    \item Note the placement of periods in the following abbreviations:
    \begin{quote}
    e.g. `for example, \dots'\\
    i.e. `that is, \dots'\\
    cf. `compare'
    \end{quote}
    These abbreviations should always precede the thing they are modifying.
    \item If not followed by a comma or colon, non--sentence-final abbreviations including
    \emph{e.g.}, \emph{i.e.}, \emph{cf.}, \emph{p.}, \emph{vs.}, and \emph{etc.}\
    should be separated from the next word in the sentence by a tilde (\texttt{\tat}) or backslash+space.\footnote{The difference between these is that the tilde prevents a line break. The tilde is ideal for \emph{e.g.}, \emph{i.e.}, \emph{cf.}, and \emph{p.}, which are essentially prefixes and should not appear at the end of a line.} This is to prevent \LaTeX{} from thinking it's the end of a sentence and adding a bit of extra space.\footnote{Because an end-of-sentence period may be followed by a close parenthesis or close quote, this policy applies for parenthesized or quoted ``abbrevs.''\ as well.} Compare:

    \begin{quote}
    E.g. preps. (and postps.) are the best (UNEVENLY SPACED)

    E.g.~preps.\ (and postps.)\ are the best (EVENLY SPACED)
    \end{quote}
    \item Dashes are wider than hyphens and serve a different function.
    For numeric ranges, use \verb|--|, which typesets an \emph{en~dash}.
    For phrase separation similar to a colon or parentheses, use \verb|---|,
    which typesets an \emph{em~dash}. In \verb|stroop.sty| these are not separated by spaces:

    \begin{quote}
    pp.~5--10
    \end{quote}
    \begin{quote}
    ADPs---e.g., \p{in}, \p{at}, and \p{for}---are highly frequent in English.
    \end{quote}
    \item For textual lists of three or more items, StroopNLP uses the serial (Oxford) comma before the conjunction, as in the above example.
    \item If not in an equation or formula, typeset the numerical part outside of math mode, but numerical modifier symbols \emph{in} math mode to distinguish them from textual punctuation:

    \begin{quote}
    between $-$86.3 and $+$92.7
    \end{quote}

    Use math symbol \verb|\approx|, not \verb|~|, for approximate quantities:

    \begin{quote}
    $\approx$100 minutes
    \end{quote}
\end{itemize}

See also the \href{http://cljournal.org/style_guide_general.html}{style guide} for the \emph{Computational Linguistics} journal.

\subsection{Math}

\begin{itemize}
\item Never say something is obvious, clear, etc.
\item Equations are a part of text: always end them with the correct
punctuation. For full-line equation, leave a small space before the
punctuation, like this:
\[ \sigma(t) = \frac{1}{1+\exp(-t)}\,. \]
\item Never write words in math mode without wrapping them in one of
\verb|\text|,
\verb|\operatorname|,
\verb|\mathrm|,
or similar.
In particular, don't use math mode for
italic effect, use \verb|\textit|:
\[ \text{wrong: } x_{function} \qquad \text{right: } x_\textit{function}\,. \]
Notice the spacing around the \emph{f} and \emph{t}.
\LaTeX{} interprets a string like \verb|$function$| as the product of
a bunch of variables $f \times u \times n \times \ldots$.
The correct way to typeset words depends on the semantics:
\begin{itemize}
\item Actual narrative text should be written inside \verb|\text|,
which inherits the properties of the surrounding environment. For example:

\[ p_i = \frac{\exp x_i}{Z}\,,
\quad \text{where} \quad
Z=\sum_i \exp x_i\,.\]

\begin{theorem}
The normalized probabilities are given by
\[ p_i = \frac{\exp x_i}{Z}\,,
\quad \text{where} \quad
Z=\sum_i \exp x_i\,.\]
\end{theorem}
\item Operators like $\exp$ should use builtins if available, or
otherwise, should be declared with \verb|\operatorname| and,
if repeated more than once, should be declared in the preamble using
\verb|\DeclareMathOperator|:
\[\bm{p} = \operatorname{softmax}(\bm{x})\,.\]
\item Other objects that should be typeset as text should use \verb|\mathrm|,
for instance the volume unit, \eg, $\mathrm{d}x$,
or units of measurement, \eg, $h=28\mathrm{cm}$.
However, for the volume unit specifically, \verb|stroop.sty|
provides \verb|\dif| which better handles spacing;
for units, use the package \verb|siunitx|.
\end{itemize}

\item \Citet{typesetting-math} provide key tips for how to use mathematical notation
in NLP and machine learning.
There is also an overview of available math symbols and environments  \citep{downes-17}.
The advice here, however, takes precedence when overlapping. (For instance,
\citet{typesetting-math} wrongly suggests typesetting operator names using \verb|\mathrm| instead of \verb|\OperatorName|,
which leads to slightly incorrect spacing.
\end{itemize}

Alphabet example:

\[ \bm{u} \equiv (\bm{x}, x) \in \cS \coloneqq \bbR^d \times \bbZ \]

\subsection{Tables and figures}

\begin{itemize}
    \item Contents of tables and figures can use small font.
    \item Table and figure captions should be informative, explaining important abbreviations, measures, and experimental conditions.
    \item Tables should minimize visual distraction due to rules between rows and columns; usually vertical rules are unnecessary and a bit of extra space between groups of rows can suffice. See \cref{tab:numtbl}.
    \item When entering numbers into a table, take care that they are aligned flush right and use a consistent number of digits after the decimal point.
    Usually percent (\%) signs are unnecessary in the table provided that it is clear what the columns mean.\\ \hspace*{10pt} The column header should be centered relative to the numbers.
    For a wide column whose values have different numbers of digits, you can fake this by centering the column and specifying \verb|\hphantom{0}| or similar before shorter values (\cref{tab:numtbl}).
    \item Column specifier \texttt{H} will create a hidden column in a \texttt{tabular}.
\end{itemize}

\begin{table}
    \centering\small
    \begin{tabular}{lrcH}
    \textbf{Method} & \multicolumn{1}{c}{$F_1$} & \textbf{OOV Recall} \\
      \midrule
        Baseline & 66.3 & \hphantom{1}9.5 & xxx-I-am-hidden-xxx \\
        SVM & 68.0 & \hphantom{1}9.8 \\
        MLP & 72.2 & 14.0 \\[3pt]
        Rule-based system (RB) & 56.2 & \hphantom{1}9.1 \\
        RB + SVM & 70.9 & \textbf{18.5} \\
        RB + MLP & \textbf{74.8} & 18.4 \\
    \end{tabular}
    \caption{Dev set performance of systems using default hyperparameters: overall $F$-score percentages and out-of-vocabulary word recall percentages. The highest value in each column is bolded.}
    \label{tab:numtbl}
\end{table}

\section{Graphics}


Wherever possible, images should be encoded as \href{https://simple.wikipedia.org/wiki/Vector_graphics}{vector graphics}
(typically achieved by exporting to PDF or SVG rather than PNG, TIFF, GIF, or JPEG) to allow for resizing and zooming without loss of resolution.
Graphics can also be generated with \LaTeX{} packages such as \href{https://en.wikibooks.org/wiki/LaTeX/PGF/TikZ}{TikZ}.
Color palettes should be colorblind-friendly (e.g., \href{https://personal.sron.nl/~pault/data/colourschemes.pdf}{Paul Tol's color schemes}).

\section{Writing}

\subsection{Word choice}

The expressions in \cref{tab:conv} sound highly conversational or ungrammatical and should be avoided in academic writing.
The expressions in \cref{tab:wrongimpression} should be used to avoid giving the reader the wrong impression.

\begin{table}[t]
\begin{center}\small
\begin{tabular}{>{\raggedright}p{9em}>{\raggedright}p{12em}H}
\toprule
\textbf{Avoid} & \textbf{Instead use} & \\
\midrule
\tablehead{\toprule
\textbf{Avoid} & \textbf{Instead use} & \\
}
\tabletail{\bottomrule}
a lot of & much \newline many \newline considerable \newline a great deal of & \\
\midrule
Besides, & In addition, \newline Additionally, & \\
\midrule
Following, & Next, \newline Additionally, \newline \emph{(OK after ``the'': “the following contributions”)} & \\
\midrule
nowadays \newline these days \newline lately & currently \newline recently \newline \emph{(or rephrase with an adjective: “current/contemporary approaches”)} & \\
\midrule
allows to & allows us to \newline permits \newline facilitates & \\
\midrule
many works & much work \newline many studies & \\
\midrule
in this research, we\dots & in this paper\slash study\slash work\newline (\emph{``research'' usually refers to a body of research})
& \\
\bottomrule
\end{tabular}
\end{center}
\caption{\label{tab:conv}Phrasing to avoid conversational style.}
\end{table}


\begin{table}[t]
\begin{center}\small
\begin{tabular}{>{\raggedright}p{10.5em}>{\raggedright}p{10.516em}H}
\toprule
\textbf{Avoid} & \textbf{Instead use} & \\
\midrule
we prove that \newline \emph{(if no mathematical proof)} & we show\slash demonstrate\slash argue\slash offer evidence that & \\
\midrule
significantly improves\newline significant improvement\newline significantly better \newline \emph{(if no test showing statistical significance)} & substantially improves \newline greatly improves \newline substantial improvement \newline substantially better & \\
\bottomrule
\end{tabular}
\end{center}
\caption{\label{tab:wrongimpression}Phrasing to avoid giving the reader the wrong impression.}
\end{table}


\subsection{Organization and other considerations}

\begin{itemize}
\item Is That Footnote Really Necessary?

\item \Citeauthor{writing-well}'s presentation is an excellent starting point
for advice on how to write a clear and effective paper.

\end{itemize}

\subsection{Algorithms and Code}\label{sec:algo}

\begin{algorithm}[t]
\begin{algorithmic}[1]
\STATE $i\gets 10$
\IF {$i\geq 5$}
  \STATE $i\gets i-1$
  \STATE $i\gets i^2$ \COMMENT{hi}
\ELSE
  \IF {$i\leq 3$}
    \STATE $i\gets i+2$
  \ENDIF
\ENDIF
\end{algorithmic}
\caption{Example algorithm. \label{algo:rhythm}}
\end{algorithm}
